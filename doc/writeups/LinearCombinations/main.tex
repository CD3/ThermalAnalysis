\documentclass[letterpaper,10pt]{article}
\usepackage{amsmath}
\usepackage{amsfonts}
\usepackage{amssymb}
\usepackage{graphicx}
\usepackage{siunitx}
\usepackage{physics}
\usepackage[shell,siunitx]{gnuplottex}
\usepackage[left=1in,right=1in,top=1in,bottom=1in]{geometry}
\DeclareMathOperator{\erfc}{erfc}

\begin{document}

\section{Introduction}
\label{sec:introduction}



The heat equation,
\begin{equation}
  \rho c \pdv{T}{t} = \div{\kappa\grad{T}} + A,
\end{equation}
is linear. This means that any solutions to the heat equation can be added together, and will also
be solutions to the heat equation. I.e., if
\begin{align*}
  \rho c \pdv{T_1}{t} &= \div{\kappa\grad{T_1}} + A_1 \\
  \qand \\
  \rho c \pdv{t}T_2 &= \div{\kappa\grad{T_2}} + A_2 \\
  \qthen \\
  \rho c \pdv{t}\qty(\alpha T_1 + \beta T_2) &= \div{\kappa\grad{\qty(\alpha T_1 + \beta T_2)}} + \alpha A_1 + \beta A_2 \\
\end{align*}
We can therfore construct the thermal response to a given source term $A$ by adding together the thermal response
from other source terms. Let
\begin{equation}
  A = \sum\limits_{i=1}^N \alpha_i A_i(t)
\end{equation}
then the thermal response $T$ for the source term $A$ can be written
\begin{equation}
  T = \sum\limits_{i=1}^N \alpha_i T_i(t)
\end{equation}
where $T_i$ is the termal response to source term $A_i$.

% section introduction (end)

\section{Applications}
\label{sec:applications}

\subsection{Multiple-Pulse Exposures}
\label{sub:multiple_pulse_exposures}



Simulating the thermal response to a multiple-pulse exposure can be expensive, and it is often (much) faster
to build the thermal response to a multiple-pulse exposure by simulating a single-pulse exposure and using
it to build the moutiple-pulse thermal response.

Consider an exposure to $N$ identical pulses with duration $\tau$ and period $t_0$. First, run a heat solver
to compute $T_\tau(t)$, the termal response to single pulse with source term
\begin{equation}
  A_\tau(t) = \begin{cases}
    0 & t < 0 \\
    A_0 & 0 \le t \le \tau \\
    0 & t > \tau
  \end{cases}.
\end{equation}
Then construct $T(t)$,
\begin{equation}
  T(t) = \sum\limits_{i = 0}^{N-1} T_\tau(t - it_0)
\end{equation}


% subsection multiple_pulse_exposures (end)

\subsection{Single-Pulse Exposures}
\label{sub:single_pulse_exposure}

The linearity of the heat equation can also be used to construct \emph{single}-pulse thermal responds. Consider
the thermal response to an infinitely long source term,
\begin{align}
  T_\infty(t) \\
  \text{with} \\
  A_\infty(t) = \begin{cases}
    0 & t < 0 \\
    A_0 & t \ge 0
  \end{cases}
\end{align}
Then the source term for a $\tau$ long exposure can be written
\begin{equation}
  A_\tau(t) = A_\infty(t) - A_\infty(t - \tau)
\end{equation}
and the thermal response will be
\begin{equation}
  T_\tau(t) = T_\infty(t) - T_\infty(t - \tau)
\end{equation}
This can be useful if one needs the thermal response for several different exposure durations. A heat solver
only needs to be ran once, and the thermal responses can be constructed afterward. Note that the heat solver
will need to be ran for a sufficiently long time.

The other potential benefit of this approach is that it can be used when an analytical solution for $T_\infty$
exists, but a general solution does not. For example, the one-dimensional cartesian coordinate
Green's Function for an infinite homogenious media is
\begin{equation}
  \label{eq:1d_cart_green_func}
  G(x,x^\prime,t,t^\prime) = \frac{1}{\sqrt{4\pi k (t-t^\prime)}} \exp\qty(\frac{-(x-x^\prime)^2}{4k(t-t^\prime)})
\end{equation}
where $k$ is the thermal diffusivity, $k = \frac{\kappa}{\rho c}$. Green's Functions are solutions to a
partial differential equation for a delta-function source term and satisfy the boundary conditions.
In this case, $A = \delta(x-x^\prime)\delta(x-x^\prime)$ and $T(x = \pm \infty,t) = 0$. The solution
to a general source term $A(x,t)$ can then be written as a convolution of the Green's Function with the source
term
\begin{equation}
  \label{eq:1d_cart_green_func_conv}
  T(x,t) = \int_0^t \int_{-\infty}^{\infty} G(x,x^\prime,t,t^\prime) A(x^\prime,t^\prime) \dd x^\prime \dd t^\prime.
\end{equation}

Now consider infintesimal source on the $x = 0$ plane that turns on at $t = 0$ and remains on. The temperature
at some position $x$ will be
\begin{equation}
  \label{eq:1d_cart_delta_func_source}
  T(x,t) = \int_0^t  \frac{1}{\sqrt{4\pi k (t-t^\prime)}} \exp\qty(\frac{-x^2}{4k(t-t^\prime)}) \dd t^\prime.
\end{equation}
This is not an analytical solution, but it can be numerically integrated, which
can be faster (and more accurate) than a numerical PDE solver. At $x = 0$, this reduces to
\begin{equation}
  T(x,t) = \int_0^t  \frac{1}{\sqrt{4\pi k (t-t^\prime)}}\dd t^\prime =
  \left. \frac{-2 \sqrt{t - t^\prime}}{\sqrt{4\pi k}} \right |_0^t =
    \sqrt{ \frac{t}{\pi k}}
  \end{equation}
  The temperature response to a \SI{4}{\second} exposure is then (See Figure \ref{fig:Tvst})
  \begin{equation}
    T(x=0,t) = \sqrt{ \frac{t}{\pi k}} - \sqrt{ \frac{t - 4}{\pi k}}
  \end{equation}

  \begin{figure}
    \begin{gnuplot}
      set xlabel "time (s)"
      set key left
      set xrange[0:10]
      T(t) = t > 0 ? sqrt(t/pi) : 0
      plot T(x) - T(x-4)
    \end{gnuplot}
    \caption{\label{fig:Tvst} }
  \end{figure}

  % subsection single_pulse_exposure (end)

  \subsection{Spatial Combinations}
  \label{sub:spatial_combinations}

  The heat equation is linear in space too, and we can build thermal distributions with linear combinations as well.
  Consider a Beer's Law absorber, with the surface positioned at $x = 0$. The source term will be given by Beer's Law
  \begin{equation}
    \label{eq:beers_law}
    A(x) = \begin{cases}
      0 & x < 0 \\
      \mu_a E_0 e^{-\mu_a x} & x \ge 0
    \end{cases}
  \end{equation}
  Plugging this into Equation \ref{eq:1d_cart_green_func_conv} gives
  \begin{equation}
    T(x,t) = \int_0^t \int_{0}^{\infty}
    \frac{1}{\sqrt{4\pi k (t-t^\prime)}} \exp\qty(\frac{-(x-x^\prime)^2}{4k(t-t^\prime)})
    \mu_a E_0 e^{-\mu_a x^\prime}
    \dd x^\prime \dd t^\prime.
  \end{equation}
  which can be rearanged,
  \begin{equation}
    T(x,t) =
    \frac{\mu_a E_0}{\sqrt{4\pi k}}
    \int_0^t
    \frac{1}{\sqrt{(t-t^\prime)}}
    \int_{0}^{\infty}
    \exp\qty(\frac{-(x-x^\prime)^2}{4k(t-t^\prime)}-\mu_a x^\prime)
    \dd x^\prime \dd t^\prime.
  \end{equation}
  The exponent can be expanded
  \begin{align}
    \frac{-(x-x^\prime)^2}{4k(t-t^\prime)}-\mu_a x^\prime
&=
\frac{-x^{\prime 2} + 2xx^\prime - x^2}{4k(t-t^\prime)}-\mu_a x^\prime \\
&=
\frac{-1}{4k(t-t^\prime)} x^{\prime 2}
+
\qty(\frac{2x}{4k(t-t^\prime)} -\mu_a) x^{\prime}
+
\frac{-1}{4k(t-t^\prime)} x^{2}
\end{align}
Which gives
\begin{equation}
  T(x,t) =
  \frac{\mu_a E_0}{\sqrt{4\pi k}}
  \exp\qty(\frac{-x^{2}}{4k(t-t^\prime)} )
  \int_0^t
  \frac{1}{\sqrt{(t-t^\prime)}}
  \int_{0}^{\infty}
  \exp\qty(
  \frac{-1}{4k(t-t^\prime)} x^{\prime 2}
  +
  \qty(\frac{2x}{4k(t-t^\prime)} -\mu_a) x^{\prime}
  )
  \dd x^\prime \dd t^\prime.
\end{equation}
Let
\begin{align}
  a &=
  \frac{1}{4k(t-t^\prime)} \\
  b &=
  \mu_a - \frac{2x}{4k(t-t^\prime)}
\end{align}
Then the spatial integral can be written
\begin{equation}
  \label{eq:spatial_int}
  \int_{0}^{\infty}
  \exp\qty(
  -a x^{\prime 2}
  -b x^{\prime}
  )
  \dd x^\prime
\end{equation}
If we complete the square, this can be written as a Gaussian in $x^\prime$.
\begin{align}
  (Ax^\prime + B)^2 &= A^2x^{\prime 2} + 2ABx^\prime + B^2 \\
  A^2 &= a \\
  2AB &= b \\
  ax^{\prime 2} + bx^\prime + B^2 - B^2
      &=
      ax^{\prime 2} + bx^\prime + \frac{b^2}{4a} - \frac{b^2}{4a} \\
      &=
      \qty(\sqrt{a}x^\prime + \frac{b}{2\sqrt{a}})^2 - \frac{b^2}{4a}
    \end{align}
    Now the spatial integral can be written as
    \begin{equation}
      \int_{0}^{\infty}
      \exp \qty(-\qty(\sqrt{a}x^\prime + \frac{b}{2\sqrt{a}})^2 + \frac{b^2}{4a})
      \dd x^\prime
      =
      \exp \qty(\frac{b^2}{4a} )
      \int_{0}^{\infty}
      \exp \qty(-\qty(\sqrt{a}x^\prime + \frac{b}{2\sqrt{a}})^2)
      \dd x^\prime
    \end{equation}
    Let
    \begin{equation}
      u
      =
      \sqrt{a}x^\prime + \frac{b}{2\sqrt{a}}
    \end{equation}
    Then
    \begin{equation}
      \dd u
      =
      \sqrt{a} \dd x^\prime
    \end{equation}
    and
    \begin{equation}
      \int_{0}^{\infty}
      \exp \qty(-\qty(\sqrt{a}x^\prime + \frac{b}{2\sqrt{a}})^2)
      \dd x^\prime
      =
      \int_{\frac{b}{2\sqrt{a}}}^{\infty}
      \exp \qty(-u^2)
      \frac{1}{\sqrt{a}}\dd u
    \end{equation}
    Recall that the Error function is defined as
    \begin{equation}
      \erf(x) = \frac{2}{\sqrt{\pi}} \int_0^x e^{-t^2} dt.
    \end{equation}
    and the complementary error function is defined as
    \begin{equation}
      \erfc(x) = 1 - \erf(x) = \frac{2}{\sqrt{\pi}} \int_x^\infty e^{-t^2} dt.
    \end{equation}

    which gives
    \begin{align}
      \frac{1}{\sqrt{a}}
      \int_{\frac{b}{2\sqrt{a}}}^{\infty}
      \exp \qty(-u^2)
      \dd u
  &=
  \frac{1}{2} \sqrt{\frac{\pi}{a}}
  \erfc\qty( \frac{b}{2\sqrt{a}} ).
\end{align}
Equation \ref{eq:spatial_int} can be evaluated then
\begin{equation}
  \int_{0}^{\infty}
  \exp\qty(
  -a x^{\prime 2}
  -b x^{\prime}
  )
  \dd x^\prime
  =
  \exp \qty(\frac{b^2}{4a} )
  \frac{1}{2} \sqrt{\frac{\pi}{a}}
  \erfc\qty( \frac{b}{2\sqrt{a}} ).
\end{equation}
We now substitue back back for $a$ and $b$. Note that
\begin{align}
  \label{eq:beers_law_temp}
  \frac{1}{\sqrt{a}} &= 2\sqrt{k(t-t^\prime)} \\
  \frac{b}{2\sqrt{a}} &= \qty(\mu_a - \frac{2x}{4k(t-t^\prime)})\sqrt{k(t-t^\prime)} \\
                      &= \mu_a \sqrt{k(t-t^\prime)} - \frac{x}{2\sqrt{k(t-t^\prime)}} \\
                      \frac{b^2}{4a} &= \mu_a^2 k(t-t^\prime) + \frac{x^2}{4k(t-t^\prime)} - \mu_a x
                    \end{align}
                    which gives
                    \begin{align}
                      T(x,t) &=
                      \frac{\mu_a E_0}{\sqrt{4\pi k}}
                      \exp\qty(\frac{-x^{2}}{4k(t-t^\prime)} )
                      %
                      \int_0^t
                      \frac{1}{\sqrt{(t-t^\prime)}}
                      \exp \qty(\frac{b^2}{4a} )
                      \frac{1}{2} \sqrt{\frac{\pi}{a}}
                      \erfc\qty( \frac{b}{2\sqrt{a}} )
                      %
                      \dd t^\prime \\
                      %
                      %
         &=
         \frac{\mu_a E_0}{\sqrt{4\pi k}}
         \exp\qty(- \mu_a x)
         %
         \int_0^t
         \frac{
           \exp\qty( \mu_a^2 k(t-t^\prime) )
           }{
           \sqrt{(t-t^\prime)}
         }
         \sqrt{\pi k (t - t^\prime)}
         \erfc\qty(
         \mu_a \sqrt{k(t-t^\prime)} - \frac{x}{2\sqrt{k(t-t^\prime)}}
         )
         \dd t^\prime \\
         &=
         \frac{\mu_a E_0}{2}
         \exp\qty(- \mu_a x)
         %
         \int_0^t
         \exp\qty( \mu_a^2 k(t-t^\prime) )
         \erfc\qty(
         \mu_a \sqrt{k(t-t^\prime)} - \frac{x}{2\sqrt{k(t-t^\prime)}}
         )
         \dd t^\prime.
       \end{align}
Evaluating the temporal integral, which can be done numerically, gives the temperature distribution for a linear absorber embedded
in a thermally homogeneous media, exposed to an infinitely large beam. The response to a \emph{finite} thickness absorber can be constructed
as a linear combination of this solution. Consider linear absorber of thickness $d$. The source term for this absorber will be
\begin{equation}
  \label{eq:beers_law_finite}
  A(x) = \begin{cases}
    0 & x < 0 \\
    \mu_a E_0 e^{-\mu_a x} & 0 \le x \le d \\
    0  & x > d
  \end{cases}
\end{equation}
which can be written as a linear  combination of two infinite thickness absorbers
\begin{equation}
  A(x) = A_\infty(x) - e^{-\mu_a d}A_\infty(x-d)
\end{equation}
where $A_\infty$ is the source term in Equation \ref{eq:beers_law}. We can use the same linear combination to construct the thermal
response. A finite thickness linear absorber embedded in a homogeneous material is a commonly used to model laser exposure to the retina, so this
thermal response would model an exposure to a very large irradiated area.

Once the thermal response for a finite thickness exposure has been computed, we can construct the response to a finite duration exposure, and then
compute the response to a multiple-pulse exposure.

\subsubsection{Verification}
\label{sub:verification}
For very large absorption coefficient, Equation \ref{eq:beers_law_temp} should reduce to Equation \ref{eq:1d_cart_delta_func_source}, as a larger
absorption coefficient will cause more energy to be absorbed near the surface. To verify this,
we would need to evaluate the limit as $\mu_a \rightarrow \infty$.



% subsection spatial_combinations (end)

% section applications (end)
\end{document}

