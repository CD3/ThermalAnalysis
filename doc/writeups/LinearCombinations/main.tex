\documentclass[letterpaper,10pt]{article}
\usepackage{amsmath}
\usepackage{amsfonts}
\usepackage{amssymb}
\usepackage{graphicx}
\usepackage{siunitx}
\usepackage{physics}
\usepackage[shell,siunitx]{gnuplottex}
\usepackage[left=1in,right=1in,top=1in,bottom=1in]{geometry}

\begin{document}

\section{Introduction}
\label{sec:introduction}



The heat equation,
\begin{equation}
  \rho c \pdv{T}{t} = \div{\kappa\grad{T}} + A,
\end{equation}
is linear. This means that any solutions to the heat equation can be added together, and will also
be solutions to the heat equation. I.e., if
\begin{align*}
  \rho c \pdv{T_1}{t} &= \div{\kappa\grad{T_1}} + A_1 \\
  \qand \\
  \rho c \pdv{t}T_2 &= \div{\kappa\grad{T_2}} + A_2 \\
  \qthen \\
  \rho c \pdv{t}\qty(\alpha T_1 + \beta T_2) &= \div{\kappa\grad{\qty(\alpha T_1 + \beta T_2)}} + \alpha A_1 + \beta A_2 \\
\end{align*}
We can therfore construct the thermal response to a given source term $A$ by adding together the thermal response
from other source terms. Let
\begin{equation}
  A = \sum\limits_{i=1}^N \alpha_i A_i(t)
\end{equation}
then the thermal response $T$ for the source term $A$ can be written
\begin{equation}
  T = \sum\limits_{i=1}^N \alpha_i T_i(t)
\end{equation}
where $T_i$ is the termal response to source term $A_i$.

% section introduction (end)

\section{Applications}
\label{sec:applications}

\subsection{Multiple-Pulse Exposures}
\label{sub:multiple_pulse_exposures}



Simulating the thermal response to a multiple-pulse exposure can be expensive, and it is often (much) faster
to build the thermal response to a multiple-pulse exposure by simulating a single-pulse exposure and using
it to build the moutiple-pulse thermal response.

Consider an exposure to $N$ identical pulses with duration $\tau$ and period $t_0$. First, run a heat solver
to compute $T_\tau(t)$, the termal response to single pulse with source term
\begin{equation}
A_\tau(t) = \begin{cases}
  0 & t < 0 \\
  A_0 & 0 \le t \le \tau \\
  0 & t > \tau
\end{cases}.
\end{equation}
Then construct $T(t)$,
\begin{equation}
  T(t) = \sum\limits_{i = 0}^{N-1} T_\tau(t - it_0)
\end{equation}


% subsection multiple_pulse_exposures (end)

\subsection{Single-Pulse Exposures}
\label{sub:single_pulse_exposure}

The linearity of the heat equation can also be used to construct \emph{single}-pulse thermal responds. Consider
the thermal response to an infinitely long source term,
\begin{align}
  T_\infty(t) \\
  \text{with} \\
  A_\infty(t) = \begin{cases}
    0 & t < 0 \\
    A_0 & t \ge 0
  \end{cases}
\end{align}
Then the source term for a $\tau$ long exposure can be written
\begin{equation}
  A_\tau(t) = A_\infty(t) - A_\infty(t - \tau)
\end{equation}
and the thermal response will be
\begin{equation}
  T_\tau(t) = T_\infty(t) - T_\infty(t - \tau)
\end{equation}
This can be useful if one needs the thermal response for several different exposure durations. A heat solver
only needs to be ran once, and the thermal responses can be constructed afterward. Note that the heat solver
will need to be ran for a sufficiently long time.

The other potential benefit of this approach is that it can be used when an analytical solution for $T_\infty$
exists, but a general solution does not. For example, the one-dimensional cartesian coordinate
Green's Function for an infinite homogenious media is
\begin{equation}
  G(x,x^\prime,t,t^\prime) = \frac{1}{\sqrt{4\pi k (t-t^\prime)}} \exp\qty(\frac{-(x-x^\prime)^2}{4k(t-t^\prime)})
\end{equation}
where $k$ is the thermal diffusivity, $k = \frac{\kappa}{\rho c}$. Green's Functions are solutions to a
partial differential equation for a delta-function source term and satisfy the boundary conditions.
In this case, $A = \delta(x-x^\prime)\delta(x-x^\prime)$ and $T(x = \pm \infty,t) = 0$. The solution
to a general source term $A(x,t)$ can then be written as a convolution of the Green's Function with the source
term
\begin{equation}
  T(x,t) = \int_0^t \int_{-\infty}^{\infty} G(x,x^\prime,t,t^\prime) A(x^\prime,t^\prime) \dd x^\prime \dd t^\prime.
\end{equation}

Now consider infintesimal source on the $x = 0$ plane that turns on at $t = 0$ and remains on. The temperature
at some position $x$ will be
\begin{equation}
  T(x,t) = \int_0^t  \frac{1}{\sqrt{4\pi k (t-t^\prime)}} \exp\qty(\frac{-x^2}{4k(t-t^\prime)}) \dd t^\prime.
\end{equation}
This is not an analytical solution, but it can be numerically integrated, which
can be faster (and more accurate) than a numerical PDE solver. At $x = 0$, this reduces to
\begin{equation}
  T(x,t) = \int_0^t  \frac{1}{\sqrt{4\pi k (t-t^\prime)}}\dd t^\prime = 
  \left. \frac{-2 \sqrt{t - t^\prime}}{\sqrt{4\pi k}} \right |_0^t = 
  \sqrt{ \frac{t}{\pi k}}
\end{equation}
The temperature response to a \SI{4}{\second} exposure is then (See Figure \ref{fig:Tvst})
\begin{equation}
  T(x=0,t) = \sqrt{ \frac{t}{\pi k}} - \sqrt{ \frac{t - 4}{\pi k}}
\end{equation}

\begin{figure}
\begin{gnuplot}
  set xlabel "time (s)"
  set key left
  set xrange[0:10]
  T(t) = t > 0 ? sqrt(t/pi) : 0
  plot T(x) - T(x-4)
\end{gnuplot}
\caption{\label{fig:Tvst} }
\end{figure}



% subsection single_pulse_exposure (end)

% section applications (end)
\end{document}

